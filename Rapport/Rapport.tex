\documentclass{article}

\usepackage{amsfonts}

\author{Médéric Bouchard 20187931\\
Zi Hang Yin 20252023}

\date{8 décembre 2023}

\title{IFT3913 - TP4}

\begin{document}
\maketitle

\section{Tests boîte noire}

Pour effectuer les tests de boîte noire d'après la spécification, nous avons besoin de 3 types de données: le montant ($M$), la devise originale ($D_o$) et la devise vers laquelle on veut convertir ($D_c$).
Il faut ensuite trouver les limites de ces données et en faire les permutations pour obtenir nos classes d'équivalence.

$$M_1 = \{m < 0\}$$
$$M2 = \{0 \leq m \leq 1 000 000\}$$
$$M_3 = \{m > 1 000 000\}$$

Les classe pour $D_o$ et $D_c$ sont les mêmes:

$$D_1 = \{USD, CAD, GBP, EUR, CHF, AUD\}$$
$$D_2 = \overline{D_1}$$

Cela nous donne un jeu de test comme suit

$$T = $$

\section{Tests boîte blanche}

\end{document}