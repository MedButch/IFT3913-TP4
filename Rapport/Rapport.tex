\documentclass{article}

\usepackage{amsfonts}

\author{Médéric Bouchard 20187931\\
Zi Hang Yin 20252023}

\date{8 décembre 2023}

\title{IFT3913 - TP4}

\begin{document}
\maketitle

\section{Tests boîte noire}

Pour effectuer les tests de boîte noire d'après la spécification, nous avons besoin de 3 types de données: le montant ($M$), la devise originale ($D_o$) et la devise vers laquelle on veut convertir ($D_c$).
Il faut ensuite trouver les limites de ces données et en faire les permutations pour obtenir nos donnéees du jeu de test.

$$M_1 = \{m < 0\}$$
$$M2 = \{0 \leq m \leq 1 000 000\}$$
$$M_3 = \{m > 1 000 000\}$$

Les classe pour $D_o$ et $D_c$ sont les mêmes:

$$D_1 = \{USD, CAD, GBP, EUR, CHF, AUD\}$$
$$D_2 = \overline{D_1}$$

Cela nous donne un jeu de test comme suit

$$T = \{(M, D_o, D_c)\}$$
$$= \{(-5000, USD, CAD), (-1, CAD, GBP), (0, GBP, EUR), (500000, EUR, CHF),$$
$$(1000000, CHF, AUD), (1000001, AUD, USD), (1234567890, CAD, USD),$$
$$(7890, PES, NZD), (567, FRC, USD), (98765, CAD, MRK), (12345, CAD, CAD)\}$$

Les valeurs sont choisies de manière à représenter les valeurs typiques et les bornes du montant à échanger, tout en ayat des valeurs de dévises valides.
Ensuite, 3 cas représentant les permutations de dévises non-valides.
Finalement, nous avons rajouter un cas où on convertit d'une dévise à elle-même, où on devrait retrouver la même valeur.

Mais cela ne s'applique qu'au cas de \textit{MainWindow.convert}.
Pour \textit{Currency.convert}, c'est simplement $M$ et le taux d'échange $T_e$ qui sont nécessaire.

Donc pour \textit{Currency.convert}, le jeu de donnée sera

$$T = \{(M, T_e\})$$
$$= \{(-5000, 1.25), (-1, 1.25), (0, 1.25), (500000, 1.25), (1000000, 1.25), (1000001, 1.25), (1234567890, 1.25)\}$$

Pour tout les cas non-valides, on s'attend à une valeur indiquant une erreur, par exemple -1.

\section{Tests boîte blanche}

\end{document}